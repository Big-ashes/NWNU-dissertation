% !TeX root = ./main.tex
\ashsetup{
  % title                = {智能通信网络中深度强化学习驱动的 RIS 辅助无线资源优化研究v\bigashversion},
  title                = {智能通信网络中深度强化学习驱动的 RIS 辅助无线资源优化研究},
  title*               = {A Study on Deep Reinforcement Learning-Driven Wireless Resource Optimization in RIS-Assisted Intelligent Communication Networks},
  author               = {常恒},
  author*              = {Chang Heng}, 
  search               = {智能软件服务工程},    %研究方向
  speciality           = {软件工程},    %一级学科、专业名称
  xuemajor             = {Engineering}, % 学硕可选  Engineering(软件工程)| Natural Science(计算机科学与技术)
  speciality*          = {Software Engineer},    
  % 学硕可选 Software Engineering(软件工程)|  Computer Science and Technology(计算机科学与技术)
  % 专硕可选 Electronic Information(电子信息)|  Education(职业技术教育)
  supervisor           = {贾向东~教授},
  supervisor*          = {Professor Jia Xiangdong},
  practice-supervisor  = {企业~高级工程师},
  practice-supervisor* = {Senior Engineer Jia Xiangdong},
  date                 = {2026-05-01},  % 完成时间
  professional-type    = {电子信息},    %专业学位类别
  professional-type*   = {Professional degree type},
  secret-level         = {秘密},     % 绝密|机密|秘密|控阅,注释本行则公开
  secret-level*        = {Secret},  % Top secret | Highly secret | Secret
  % secret-year          = {10},      % 保密/控阅期限
  % reviewer             = true,      % 声明页显示“评审专家签名”
  %
  % 数学字体
  % math-style           = GB,  % 可选:GB, TeX, ISO
  math-font            = xits,  % 可选:stix, xits, libertinus
}


% 加载宏包

% 定理类环境宏包
\usepackage{amsthm}

% 插图
\usepackage{graphicx}

% 三线表
\usepackage{booktabs}

% 表注
\usepackage{threeparttable}

% 跨页表格
\usepackage{longtable}

% 算法
\usepackage[ruled,noline,linesnumbered]{algorithm2e}

% SI 量和单位
\usepackage{siunitx}

\usepackage{enumitem}

% 参考文献使用 BibTeX + natbib 宏包
% 顺序编码制
\usepackage[numbers]{natbib}
\bibliographystyle{bib/bigash-numerical}

% 著者-出版年制
% \usepackage{natbib}
% \bibliographystyle{bib/bigash-authoryear}


% 参考文献使用 BibLaTeX 宏包
% \usepackage[style=bigash-numeric]{biblatex}
% \usepackage[style=bigash-authoryear]{biblatex}
% 声明 BibLaTeX 的数据库
% \addbibresource{bib/NWNU}

% 配置图片的默认目录
\graphicspath{{figures/}}

% 数学命令
\makeatletter
\newcommand\dif{%  % 微分符号
  \mathop{}\!%
  \ifash@math@style@TeX
    d%
  \else
    \mathrm{d}%
  \fi
}
\makeatother
\newcommand\eu{{\symup{e}}}
\newcommand\iu{{\symup{i}}}

% 用于写文档的命令
\DeclareRobustCommand\cs[1]{\texttt{\char`\\#1}}
\DeclareRobustCommand\env[1]{\texttt{#1}}
\DeclareRobustCommand\pkg[1]{\textsf{#1}}
\DeclareRobustCommand\file[1]{\nolinkurl{#1}}

% hyperref 宏包在最后调用
\usepackage{hyperref}
