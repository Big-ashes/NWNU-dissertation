% !Mode:: "TeX:UTF-8"
% \ustcsetup{
%   cite-style = super,
% }

\chapter{绪论}

本模板 NWNU-dissertation 是西北师范大学计算机科学与工程学院研究生学位论文 \LaTeX{}
模板, 基于 \pkg{ustcthesis}《\href{https://github.com/ustctug/ustcthesis}{中国科学技术大学研究生学位论文}》
的\LaTeX{}模板魔改而来,
按照《\href{https://jsj.nwnu.edu.cn/2454/list.htm} {计算机科学与工程学院研究生学位论文模板(2023版)}》
(以下简称《模板》)的要求编写。

Lorem ipsum dolor sit amet, consectetur adipiscing elit, sed do eiusmod tempor
incididunt ut labore et dolore magna aliqua.
Ut enim ad minim veniam, quis nostrud exercitation ullamco laboris nisi ut
aliquip ex ea commodo consequat.
Duis aute irure dolor in reprehenderit in voluptate velit esse cillum dolore eu
fugiat nulla pariatur.
Excepteur sint occaecat cupidatat non proident, sunt in culpa qui officia
deserunt mollit anim id est laborum.

这是一个示例文本,用于测试排版和布局效果。它没有任何实际意义,只是为了填充空间。
你可以使用这段文字来查看字体、间距和段落的对齐方式是否符合预期。
如果需要更长的内容,可以继续添加类似的句子,直到满足你的需求。
希望这段文字能帮助你完成设计或开发工作,祝你一切顺利!

模板的前几章是一个简单的示例,包含了一些简单的引用以及格式演示,详细的使用文档见

\section{研究背景与意义}
随着过去几年中第五代(Fifth Generation,5G)网络的成功,通信技术的迅猛发展逐渐为第六代(Sixth Generation,6G)无线通信的崛起奠定了基础


\section{国内外研究现状分析}

5G的应用推广到标准的出台,都为通信领域带来了前所未有的发展机遇,相关研究人员纷纷投入到6G与它所具有的重要功能的研究当中,通信技术的快速发展推动IoT的迅速发展,
进而推动了实时监测应用的普及与发展,包括自动驾驶、远程医疗、工业自动化等;为了更高效与更便捷的用户服务及体验,通常要求监视器上接收到的信息要尽可能地新鲜,
那么对于信息新鲜度的度量则成为新的课题,但是传统的性能指标,比如吞吐量、延迟、中断概率等都不能做到完全表征信息新鲜度。

\section{论文结构安排}
XXXXX。

对研究内容进行了全面的总结和分析\supercite{RIS-NOMA},明确指出了当前研究中存在的不
足,并对未来的研究方向和计划进行了展望\cite{huiyi}。

































